\documentclass[a4paper]{article}
\usepackage[a4paper,left=3cm,right=2cm,top=2.5cm,bottom=2.5cm]{geometry}
\usepackage{graphicx}
\usepackage[T1]{fontenc}
\usepackage[utf8]{inputenc}
\usepackage{xcolor}
\usepackage[normalem]{ulem}
\usepackage{hyperref}
\hypersetup{colorlinks, urlcolor=blue}

\title{Learning Balance Control of a Wheel-legged Robot}
\author{Noah Böckmann, Felix Weidenmüller, Lino Willenbrink}
\date{\today}

\makeatletter
\DeclareUrlCommand\ULurl@@{%
  \def\UrlFont{\ttfamily\color{blue}}%
  \def\UrlLeft{\uline\bgroup}%
  \def\UrlRight{\egroup}}
\def\ULurl@#1{\hyper@linkurl{\ULurl@@{#1}}{#1}}
\DeclareRobustCommand*\ULurl{\hyper@normalise\ULurl@}
\makeatother

\begin{document}

\maketitle
\section{Problem Description}
Wheel-legged robots combine the energy efficiency of wheels with the adaptability of legs and therefore exhibit significant potential in applications requiring agility and terrain versatility.
However, their inherently unstable structure introduces unique difficulties, especially under conditions of nonlinearities, uncertainties, and dynamic posture changes like height adjustments.
In the context of balancing a wheel-legged robot, the primary challenge lies in achieving and maintaining stability.

This problem is crucial as it directly impacts the robot's practical usability in real-world scenarios.
Without effective balance control, the robot's ability to perform more complex tasks or navigate uneven terrain is compromised, limiting its application in industry.
% Addressing these challenges can lead to breakthroughs in robotic control systems and unlock greater efficiency and functionality for wheel-legged robots.

We are especially interested in how effective this problem can be addressed by a machine learning approach, and to compare the outcome with the results presented in the original paper with the linear regulator.
Our primary aim is to stabilize the robot in the standing equilibrium position by replacing the used LQR regulator. Additionally, we allow variations/uncertainty in the center of gravity and varying the robot height by changing the leg angle.
If time permits, the implementation of a function for moving the robot would be conceivable.


\section{Relevant Work}

\section{Research Plan}




\end{document}